
\documentclass{article}


% if you need to pass options to natbib, use, e.g.:
%     \PassOptionsToPackage{numbers, compress}{natbib}
% before loading neurips_2024


% ready for submission
\usepackage[preprint]{neurips_2024}


% to compile a preprint version, e.g., for submission to arXiv, add add the
% [preprint] option:
%     \usepackage[preprint]{neurips_2024}


% to compile a camera-ready version, add the [final] option, e.g.:
%     \usepackage[final]{neurips_2024}


% to avoid loading the natbib package, add option nonatbib:
%    \usepackage[nonatbib]{neurips_2024}


\usepackage[utf8]{inputenc} % allow utf-8 input
\usepackage[T1]{fontenc}    % use 8-bit T1 fonts
\usepackage{hyperref}       % hyperlinks
\usepackage{url}            % simple URL typesetting
\usepackage{booktabs}       % professional-quality tables
\usepackage{amsfonts}       % blackboard math symbols
\usepackage{nicefrac}       % compact symbols for 1/2, etc.
\usepackage{microtype}      % microtypography
\usepackage{xcolor}         % colors


\title{AI and Mathematics}
\author{%
Satyanarayana Gajjarapu \\
AI24BTECH11009 \\
Department of Artificial Intelligence \\
\texttt{ai24btech11009@iith.ac.in} \\
}


\begin{document}


\maketitle
\section{Prof. Terence Tao}
\begin{paragraph}
\\
Terence Tao participated in \textbf{International Mathematical Olympiad} (IMO) for the 1st time at the age of 11 and won a bronze medal. He won a silver at the age of 12 and a gold at 13, becoming the youngest to win a gold in IMO. After winning gold he never participated in IMO again. He is now working as a professor in University of California, Los Angeles.
\end{paragraph}
\section{Role of AI in Mathematics, Talk by Prof. Terence Tao}
\begin{paragraph}
\\
AI is more used in the research mathematics than the competitive mathematics. Computers and machines are used to do mathematics since long time. Machines are used to do mathematics from thousand of years since Romans. Computers are used in world war 2 to compute ballistics. Tables are considered as databases of mathematics. A table used by a lot of mathematicians currently is \textbf{Online Encyclopedia of Integer Sequences} (OEIS). Recently AlphaGeometry is developed, it can solve any geometry question asked in IMO. Another type of scientific computation which is quite powerful is called SAT solvers used to solve logic puzzles. SAT solver gave a proof for Pythagorean triple problem which is unsolved by humans. 
\end{paragraph}
\begin{paragraph}
\\
Proof assistants are almost used by all mathematicians nowadays. The first computer assisted proof is for 4 colour theorem. The first complete proof by proof assistant is given by \textbf{Coq} in 2005. The proof for Kepler conjucture stated in 17th century is completed in 1996 after the assistance of a computer. \textbf{Lean} is the recent proof assistant language which has all the proofs of mathematics in its central math library. Lean can convert a very long computer proof to a human readable proof. Lean formalizes a lot of traditional mathematics.
\end{paragraph}
\begin{paragraph}
\\
\textbf{Machine Learning} uses neural networks to predict the answers to the various questions. Knot theory, an application of machine learning is a very interesting field of mathematics. In knot theory, people used machine learning and trained neural networks so that 90 percent of times it gives right signature. \textbf{Large language models} (LLMs) came into the world only 5 years ago but gives human level output. GPT - 4, one of the ChatGPT's model solved a 2022 IMO question. But when tested by hundreds of IMO questions has only 1 percent success rate. 
\end{paragraph}
\begin{paragraph}
\\
The tasks which are difficult to humans can be easy to AI and tasks easy to humans can be difficult to AI. Other than proof assistants to which entire proof should be given to get compiled, there are other applications like GitHub Copilot to which if half of the proof is given it suggests the next line. To solve narrowly focused problems we can use AI specialised programs but they are not fully reliable. In the future, a kind of AI which converts proofs from one language to another should be developed.
\end{paragraph}



\end{document}
